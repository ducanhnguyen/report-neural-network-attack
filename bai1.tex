\documentclass[12pt]{report}


% Title Page
\title{Mười quy tắc giúp nghiên cứu sinh \\hoàn thành bằng tiến sĩ liên ngành}
\author{\textit{Dịch giả}: Nguyễn Đức Anh
	\\ \textit{Nhóm tác giả}: Samuel Demharter, Nicholas Pearce, Kylie Beattie, \\Isabel Frost, Jinwoo Leem, Alistair Martin, Robert Oppenheimer, \\Cristian Regep, Tammo Rukat, Alexander Skates, Nicola Trendel, \\David J. Gavaghan, Charlotte M. Deane, Bernhard Knapp}

\setcounter{secnumdepth}{0}

\date{}
\begin{document}
\maketitle

%\begin{abstract}
	
%\end{abstract}

\textbf{Tên gốc}: Ten simple rules for surviving an interdisciplinary PhD

\textbf{Ngày đăng}: May 25, 2017

\textbf{DOI}: https://doi.org/10.1371/journal.pcbi.1005512

\textbf{Người dịch}: Nguyễn Đức Anh - Đại học Công nghệ, ĐHQGHN

\pagebreak

\tableofcontents

\pagebreak

\section{Lời nói đầu}

Khác với các nghiên cứu đơn ngành vốn chỉ giới hạn trong một lĩnh vực nghiên cứu, nghiên cứu liên ngành đòi hỏi nghiên cứu sinh phải tiếp cận các lĩnh vực nghiên cứu khác nhau. Chương trình đào tạo nghiên cứu sinh theo hướng này được gọi là chương trình đào tạo tiến sĩ liên ngành. Ví dụ, ở Mỹ có tổ chức khoa học quốc gia (NSF) \cite{1} \cite{2}; ở Châu  u có mạng lưới đào tạo tiên tiến của hội đồng nghiên cứu (ITN) \cite{3}; ở Anh có đến hội đồng nghiên cứu khoa học vật lý và kĩ thuật Anh (EPSRC) và trung tâm đào tạo tiến sĩ (CDT) \cite{4}.

Đầu tiên, đối với một nghiên cứu sinh, lựa chọn chương trình học tiến sĩ liên ngành là một cơ hội tuyệt vời để theo đuổi nghiên cứu thuộc các chủ đề khác nhau. Các cá nhân này sẽ có nhiều cơ hội nghề nghiệp hơn so với các nghiên cứu sinh đơn ngành. Tuy nhiên, nếu theo định hướng nghiên cứu liên ngành, nghiên cứu sinh sẽ phải đổi mặt với nhiều rủi ro và khó khăn hơn so với các nghiên cứu sinh theo chương trình tiến sĩ đơn ngành. Vì thế, bài viết này đề xuất 10 quy tắc để hướng dẫn một nghiên cứu sinh liên ngành đạt được mục tiêu nghiên cứu với rủi ro nhỏ nhất có thể.

Thuật ngữ “liên ngành” có rất nhiều ý nghĩa. Trong phạm vi bài viết này, “liên ngành” có nghĩa là hai lĩnh vực, hai ngành khoa học, hai chủ đề nghiên cứu, v.v. có sự liên quan mật thiết với nhau \cite{5}. Mặc dù lời khuyên cho các sinh viên quan tâm đến chương trình nghiên cứu liên ngành tương tự như đối với các cá nhân theo chương trình cao học truyền thống \cite{6}, có những điểm khác biệt mà các sinh viên định hướng học liên ngành cần nhận biết được để chuẩn bị cho chương trình học tiến sĩ liên ngành nói riêng, cũng như chương trình nghiên cứu liên ngành nói chung \cite{7}.

Hiện tại, có nhiều chương trình học tiến sĩ liên ngành, và cách tổ chức chương trình học giữa các quốc gia có điểm khác biệt nhất định tùy theo triết lý giáo dục của quốc gia đó. Ở Anh, chương trình học liên ngành được tài trợ chính bởi trung tâm đào tạo tiến sĩ (CDT). Đối với một cá nhân có nhu cầu học cao lên, việc tìm hiểu các nguồn tài trợ chính trong lĩnh vực nghiên cứu họ đang quan tâm là một điểm tựa cần thiết để biết được những chương trình liên ngành nào phù hợp với bản thân họ. Các chương trình học tiến sĩ liên nghành khác nhau có cách tổ chức khác nhau: nội dung khóa học có thể cấu trúc khác nhau, bao gồm các khóa học tiền điều kiện và các dự án ngắn hạn trước chương trình PhD; một vài chương trình cần các giai đoạn đào tạo liên tục trong suốt chương trình PhD; và một vài yêu cầu nghiên cứu sinh phải giảng dậy, trong khi một vài chương trình không yêu cầu nghiên cứu sinh phải đứng lớp. Một vài chương trình liên ngành cho phép nghiên cứu sinh bắt đầu theo học mà không cần chọn ngay giáo sư hướng dẫn. Do đó, cách tiếp cận này sẽ giúp nghiên cứu sinh dễ dàng quen thuộc với các lĩnh vực nghiên cứu khác nhau trước khi bắt đầu chọn chủ đề nghiên cứu thật sự. Đối với loại chương trình đào tạo này, việc nghiên cứu sinh phải có ngay một lĩnh vực nghiên cứu cụ thể ngay khi họ theo học không thực sự quá quan trọng nếu so sánh với chương trình đào tạo tiến sĩ truyền thống. Tuy nhiên, nếu một cá nhân lựa chọn theo chương trình tiến sĩ liên ngành, họ cần phải tìm hiểu kĩ các nhóm nghiên cứu liên quan đến chương trình đào tạo trước khi nộp đơn để đảm bảo rằng chương trình đào tạo phù hợp với mục đích theo học tiến sĩ của họ.

Chúng tôi là một nhóm các nghiên cứu sinh và giám đốc đào tạo ở trung tâm đào tạo tiến sĩ (DTC) tại trường đại học Oxford. Trung tâm DTC được thành lập vào năm 2002 và đã đào tạo hơn 550 sinh viên theo học bảy chương trình liên ngành. Rút ra từ kinh nghiệm thực tiễn, chúng tôi đề xuất 10 nguyên tắc đơn giản để giúp một nghiên cứu sinh liên ngành có thể tồn tại và sống sót. Chúng tôi sẽ tập trung nêu bật sự quan trọng của một kế hoạch nghiên cứu kĩ lưỡng cùng với một lịch trình khả thi, tầm quan trọng của việc duy trì một kênh liên lạc tốt với tất cả những người hướng dẫn và các cá nhân hợp tác, cùng với sự tự do về trí tuệ để nghiên cứu trong môi trường đào tạo tiến sĩ liên ngành.

\section{Quy tắc thứ 1: Hãy đưa tất cả mọi người vào trong kế hoạch, và lập các kế hoạch dự phòng}

Thông thường, chương trình đào tạo tiến sĩ liên ngành sẽ yêu cầu các nhân ứng tuyển cần có các kĩ năng nghiên cứu liên quan, có nhiều hơn một giáo sư hướng dẫn, và có mối liên hệ phụ thuộc vào nhiều người hợp tác. Do đó, cá nhân ứng tuyển nên đưa tất cả mọi giáo sư hướng dẫn vào kế hoạch càng sớm càng tốt. Tương tự như vậy, cá nhân theo học chương trình tiến sĩ liên ngành cần lập ra một kế hoạch có tính tương tác nhóm để hoàn thành chương trình với mục tiêu mong đợi. 

Đối với tất cả mọi nghiên cứu sinh liên ngành, việc xây dựng một biểu đồ thời gian với một nghiên cứu cụ thể là một điều quan trọng và mang lại nhiều lợi ích. Theo phương pháp này, nghiên cứu sinh sẽ đánh giá được những điều họ cần thực hiện trong suốt chương trình học tiến sĩ và nhận ra các vấn đề tiềm tàng. Hãy nhớ rằng kế hoạch này có thể thay đổi trong quá trình học sau này, vì thế, kế hoạch mà nghiên cứu sinh xây dựng cần có sự linh hoạt nhất định để đối mặt với sự thay đổi. Bên cạnh đó, các nghiên cứu sinh hãy đảm bảo có kế hoạch dự phòng, ví dụ như khi những người làm chung nhóm không chuyển giao dữ liệu hoặc bỏ cuộc giữa chừng.

Khi lên kế hoạch cho dự án, nghiên cứu sinh hãy nhìn xa hơn để chọn đúng người hướng dẫn và mạng lưới hỗ trợ phù hợp. Nghiên cứu sinh nên hỏi các giáo sư hướng dẫn để tham khảo và định hướng, đặc biệt ở giai đoạn đầu của chương trình tiến sĩ liên ngành khi nghiên cứu sinh vẫn đang tìm tòi và chưa có định hướng cụ thể.

Cuối cùng, nghiên cứu sinh cần hiểu rõ được các rào cản về hành chính. Phụ thuộc vào tính chất của bộ môn và viện nghiên cứu, sẽ có những yêu cầu khác biệt trong từng giai đoạn của chương trình PhD. Chương trình đào tạo có mức độ linh động cao có thể mang lại lợi ích và cho phép một nghiên cứu sinh đạt được yêu cầu phù hợp với họ nhất. Tuy nhiên, việc này cũng đồng nghĩa rằng nghiên cứu sinh có thể không tuân thủ đúng thủ tục được yêu cầu. Do đó, nghiên cứu sinh liên ngành cần thử lên kế hoạch năm cuối cùng của chương trình để giảm thiểu rủi ro không đáng có \cite{8}. Điều này sẽ giảm thiểu số lần nghiên cứu sinh rơi vào những sự cố không lường trước được và khiến cho những tháng cuối ít chông gai hơn. Tóm lại, một nghiên cứu sinh liên ngành cần chủ động nghĩ về những vấn đề có thể phải đối mặt.

\section{Quy tắc thứ 2: Hãy là một nhà ngoại giao bằng cách quản lý mục tiêu đạt được càng sớm càng tốt}

Nghiên cứu sinh hãy thử tổ chức một buổi gặp mặt thông thường với tất cả các giáo sư hướng dẫn và người hợp tác. Để đảm bảo tiến độ chương trình học PhD, các kế hoạch có tính tương tác và có sự thảo luận sẽ giúp giảm thiểu các tình huống không đáng tiếc, ví dụ như vấn đề không nhất trí giữa các bên liên quan về giai đoạn nghiên cứu. Bởi vì các giáo sư hướng dẫn đến từ các lĩnh vực nghiên cứu khác nhau, họ sẽ có những mục tiêu riêng đặt ra cho nghiên cứu sinh; và những mục tiêu này cần phải hoàn thành được trong một khung thời gian cụ thể. Các giáo sư có thể gặp khó khăn khi đánh giá độ khó của dự án nghiên cứu và khoảng thời gian nghiên cứu phù hợp.

Nếu các câu hỏi về hướng nghiên cứu phát sinh, ví dụ như nghiên cứu sinh cần tìm giải pháp tốt nhất cho một vấn đề, thông thường nghiên cứu sinh nên thảo luận vấn đề trực tiếp với các bên liên quan trong cùng một căn phòng.

Khi có nhiều hơn một bên liên quan trong chương trình PhD, việc giao tiếp trực tuyến như thảo luận qua skype; hoặc trao đổi trực tiếp là cần thiết.

\section{Quy tắc thứ 3: Định nghĩa lại phạm vi nghiên cứu: Khám phá và quen thuộc, rồi hãy thực dụng}

Khi bạn đã đưa tất cả những người liên quan vào trong bản kế hoạch học tiến sĩ, bạn cần bắt đầu bắt đầu nghiên cứu. Trong khi một nghiên cứu sinh truyền thống thường nhanh chóng hiểu sâu kiến thức của một chủ đề nghiên cứu hẹp thuộc về một lĩnh vực nghiên cứu cụ thể, nghiên cứu sinh liên ngành dường như mất nhiều thời gian hơn để đọc hiểu ở mức độ sâu sắc. Thay vào đó, nghiên cứu sinh liên ngành sẽ hiểu biết nhiều lĩnh vực nghiên cứu khác nhau. Do đó, việc nhận biết được và khám phá được các lĩnh vực liên quan đến chương trình PhD nên thực hiện càng sớm càng tốt. Nghiên cứu sinh nên tham gia các buổi seminar khác nhau cũng nhưng tham dự các buổi học của sinh viên để có một nền tảng vững chắc về một lĩnh vực mới, và nắm được các thuộc ngữ chuyên ngành của lĩnh vực đó. Nếu nghiên cứu sinh tập trung dành thời gian để học các môn học cơ bản từ đầu, họ sẽ được hưởng lợi ích về mặt dài hạn bởi vì quá trình tích lũy tri thức sẽ giúp họ có một bức tranh tổng thể, từ đó nhìn ra được nghiên cứu của bản thân nằm ở đâu trong bức tranh lớn đó.

Hơn nữa, chất lượng, số lượng, và cấu trúc dữ liệu có thể thay đổi giữa các lĩnh vực nghiên cứu \cite{9}. Hãy chắc chắn rằng nghiên cứu sinh cần biết điều gì họ cần, và thực hiện các quy trình kiểm tra cần thiết trên dữ liệu để đảm bảo họ có thể sử dụng chúng. Điều này sẽ cho phép nghiên cứu sinh nhận diện được các vấn đề và cho phép bản thân họ thực hiện các thay đổi cần thiết để hiểu dữ liệu một các chuẩn xác [10].

Trong khi việc quen thuộc các chủ đề liên quan đến công việc nghiên cứu là quan trọng, nghiên cứu sinh sẽ không học được mọi thứ họ cần. Trong giai đoạn học PhD, nghiên cứu sinh cần bắt đầu tập trung vào các lĩnh vực nghiên cứu chủ chốt liên quan đến nghiên cứu mà họ đang theo đuổi. Nghiên cứu sinh nên chọn phương pháp học hình phễu: bắt đầu học từ cơ bản, sau đó thu hẹp về các chủ đề nghiên cứu cụ thể. Hãy khám phá và cho mình thời gian để quen thuộc dần, nhưng nghiên cứu sinh cần thực dụng và mọi hành động cần luôn hướng đến mục tiêu.

\section{Quy tắc thứ 4: Đừng xấu hổ khi bạn hỏi những câu hỏi ngu ngốc}

Khi một nghiên cứu sinh khám giá các lĩnh vực mới, họ thông thường sẽ cảm thấy cô độc và lạc hướng. Các buổi seminar, thảo luận nhóm, nói chuyện nghiên cứu thuộc về lĩnh vực mới dường như được thực hiện ở một ngôn ngữ hoàn toàn khác. Nghiên cứu sinh thường hiếm khi tham dự các buổi gặp gỡ nơi các biệt ngữ chuyên ngành hay được sử dụng. Thực tế, nghiên cứu sinh nên dành thời gian tham gia các hoạt động này nhiều hết mức có thể, từ đó họ sẽ nhanh chóng học hỏi được nhiều thứ trong lĩnh vực mới này.

Nghiên cứu sinh liên ngành có thể mất rất nhiều thời gian để học các lĩnh vực mới một cách tự nhiên. Thông thường, tiến độ của chương trình nghiên cứu có thể rất chậm trong những tháng đầu. nghiên cứu sinh cần chủ động hỏi nhiều câu hỏi mà họ không hiểu, dù cho câu hỏi đó có đơn giản hay hiển nhiên đi chăng nữa. Họ không cần phải xấu hổ khi đặt ra những câu hỏi như vậy bởi vì những người xung quanh sẽ luôn vui vẻ giúp đỡ, và chứng kiến họ tiến bộ từng ngày. Hơn nữa, câu hỏi từ các cá nhân thuộc nền tảng khác nhau thường dẫn đến một góc nhìn mới mà chưa được xem xét bao giờ.

Tóm tắt lại, nghiên cứu sinh đừng sợ khi đặt những câu hỏi đơn giản hoặc hiển nhiên.

\section{Quy tắc thứ 5: Xây dựng mạng lưới: Tìm những người lấp đầy khoảng trống}

Bởi vì bản chất tự nhiên của nghiên cứu liên ngành, sẽ luôn có những khoảng trống trong kiến thức của bạn. Nghiên cứu sinh liên ngành cần nhận diện được các cá nhân nghiên cứu tương tự có thể trợ giúp họ ở giai đoạn đầu học PhD. Từ đó, họ có thể có một kênh tham khảo chất lượng khi cần thiết, chẳng hạn khi nghiên cứu sinh gặp trục trặc về một vấn đề nào đó. Tuy nhiên, nghiên cứu sinh liên ngành có thể không nhận được sự trợ giúp toàn diện từ các cá nhân trong bộ môn hiện tại. Vì thế, để nhận được nhiều sự trợ giúp hơn, nghiên cứu sinh liên ngành nên sử dụng công cụ tìm kiếm để tìm ra những cá nhân khác có thể trợ giúp họ, và viết một email để tìm hiểu thông thêm tin, từ đó tìm thêm được những cá nhân thực sự có thể giúp đỡ. May mắn là, hầu hết các nhà khoa học đều thân thiện và sẵn sàng chia sẻ kiến thức họ có. Nếu một nghiên cứu sinh nghi ngờ một điều gì từ sự giúp đỡ đó, giáo sư hướng dẫn có thể giúp nghiên cứu sinh kết nối với những cá nhân liên quan thay vì chính bản thân nghiên cứu sinh tự tìm.

Thiết lập và duy trì mối quan hệ với các nhà nghiên cứu có các nền tảng khác nhau là chìa khóa để hoàn thành chương trình tiến sĩ liên ngành. Việc này sẽ giúp một nghiên cứu sinh liên ngành dễ dàng hơn khi cần tìm một ai đó giải đáp câu hỏi của họ, tìm kiếm sự hợp tác thích hợp, hoặc cần một bộ dữ liệu quan trọng. Nghiên cứu sinh liên ngành hãy thử tìm những người mà họ cảm thấy thích thú khi làm việc cùng và những cá nhân có hiệu suất làm việc cao. Việc dành thời gian để phát triển các mối quan hệ như thế này là xứng đáng, bởi vì chúng có thể giúp ích rất nhiều sau giai đoạn học PhD. Một bài báo khác tập trung vào các khía cạnh khác nhau của hợp tác nghiên cứu liên ngành đã đề xuất 10 quy tắc đơn giản và có thể tìm thấy trong \cite{10}.

Khi giao tiếp với các nhà nghiên cứu ở các lĩnh vực khác nhau, nghiên cứu sinh cần chắc rằng họ có thể hiểu chính xác ý đối phương muốn truyền đạt qua thuật ngữ họ sử dụng. Các thuật ngữ tưởng chường giống nhau nhưng lại có ý nghĩa khác nhau trong các lĩnh vực nghiên cứu cụ thể. Ví dụ, từ ‘orthogonal’ trong hình học có nghĩa là hai đường thẳng trực giao; trong thống kê, các biến độc lập ảnh hưởng chung đến một biến phụ thuộc được gọi là ‘orthogonal’ nếu chúng không có sự tương quan nào; trong phân loại, một bộ phân lớp được gọi là ‘orthogonal’ nếu từng phần tử là thành viên của một nhóm duy nhất; trong hóa sinh, hai kiểu của cặp DNA có tương tác ‘orthogonal’. Nghiên cứu sinh cần hỏi và yêu cầu giải thích các thuật ngữ này nếu cần thiết.

Trong khi nghiên cứu sinh cần duy trì một vài liên lạc cần thiết, nghiên cứu sinh cũng cần thiết lập một mạng lưới kết nối lỏng lẻo nhưng đa dạng. Bạn có thể coi mạng lưới này như một mạng lưới thương gia. Nghiên cứu sinh có thể khai thác mạng lưới này khi có cơ hội đến, và họ có thể thu được lợi ích.

Nhiều cuộc hội thoại hữu ích xảy ra khi nghiên cứu sinh kì vọng nó ít nhất. Hãy thử hòa nhập với đồng nghiệp, đặc biệt ở những bối cảnh không có tính học thuật như tiệc tùng, đi du lịch, v.v.

\section{Quy tắc thứ 6: Chấp nhận bộ kĩ năng độc nhất của bạn và sử dụng chúng để định nghĩa lại phạm vi nghiên cứu}

Giá trị của mối quan hệ cộng sinh giữa các lĩnh vực nghiên cứu được nhận biết rõ ràng. Nó đến từ sự hiện thực hóa của một vài câu hỏi nghiên cứu đã bị lãng quên vì chúng không nằm rõ ràng trong phạm vi truyền thống của lĩnh vực nghiên cứu. Tương tự, các phương pháp hoặc kĩ thuật thích hợp cho một vấn đề cụ thể có thể tồn tại trong lĩnh vực khác nhưng chưa được áp dụng bởi vì tính cô lập của các lĩnh vực nghiên cứu. Nghiên cứu sinh liên ngành có ưu thế hơn nghiên cứu sinh đơn ngành bởi vì họ đang ở một vị trí thuận lợi để giải quyết các vấn đề bị bỏ bê. Bằng cách chấp nhận bộ kĩ năng nghiên cứu mà bạn đang có và tìm kiếm cơ hội để kết nối các dấu chấm, nghiên cứu sinh liên ngành có thể định nghĩa lại phạm vi lĩnh vực nghiên cứu truyền thống, hoặc tận dụng một thành tựu công nghệ từ một lĩnh vực nghiên cứu này sang lĩnh vực nghiên cứu khác.

Điều này cũng đúng khi nghiên cứu có sự cộng tác. Thông thường, các giải pháp cho các vấn đề mà một nghiên cứu sinh liên ngành đang coi là hiển nhiên có thể có giá trị cao đối với các nhà khoa học có lĩnh vực nghiên cứu khác. Nhận biết được các cơ hội này, và hãy trao đổi cởi mở góc nhìn của bạn, dù là những vấn đề đơn giản nhất. Bạn có thể ngạc nhiên về điều bạn có thể đóng góp. Đây là một cách tốt để xác lập danh tiếng giữa các đồng nghiệp, từ đó mở ra các cơ hội nghề nghiệp mới.

Tuy nhiên, hãy cẩn thận vì nghiên cứu sinh có thể dễ dàng trở thành một người cung cấp dịch vụ hay ý tưởng cho các cá nhân trong lĩnh vực khác. Họ có thể sử dụng các kĩ năng của bạn trong nghiên cứu vì họ không sở hữu các kĩ năng này. Nghiên cứu sinh có thể không muốn trở thành một người phụ trách IT cho việc tiền xử lý và tổ chức dữ liệu. Họ cũng có thể muốn tránh xa việc trở thành người chuyên phụ trách làm các thí nghiệm lặp đi lặp lại và đơn điệu. Nghiên cứu sinh chỉ nên làm những điều có lợi ích rõ ràng và có những giới hạn rõ ràng.

\section{Quy tắc thứ 7: Cảm thấy tự do khi bơi ngược dòng chảy, tiến hành thí nghiệm, và thất bại}

Nhiều lĩnh vực nghiên cứu thường có cách tiếp cận cố định với các vấn đề cụ thể. Những phương pháp tiếp cận này đã được chứng minh là đúng đắn và có một lượng lớn các nhà nghiên cứu sử dụng. Tuy nhiên, khi một nhà nghiên cứu có góc nhìn khác về các vấn đề đang gặp phải, họ có thể bị hấp dẫn bởi giải pháp mà không được coi là quy chuẩn. Đừng sợ thách thức những ý niệm cố hữu trong lĩnh vực nghiên cứu, miễn là giải pháp của bạn đã được chứng minh có những lợi thế tiềm tàng trước đó.

Hơn nữa, một vài kĩ thuật mà nghiên cứu sinh liên ngành sử dụng có thể chưa được phát triển hoặc đơn giản là chưa được áp dụng trong một lĩnh vực nghiên cứu đơn lẻ từ trước. Phát triển kĩ thuật mới, hoặc áp dụng một kĩ thuật ngoại lai, và chứng minh tính ứng dụng của chúng thường đi cùng với những rủi ro cao hơn và giải thưởng tiềm tàng cao hơn. Là một người chạy tiên phong, nghiên cứu sinh phải tự hình dung mọi thứ. Họ có thể mắc kẹt trong giai đoạn học PhD và giải pháp họ đề xuất có thể thất bại. Tuy nhiên, nghiên cứu sinh không được mất sự cảm hứng trong suốt giai đoạn nghiên cứu khó khăn này. Ngoài ra, nghiên cứu sinh có thể gặp phải sự phản đối từ giáo sư hướng dẫn khi họ đang cố gắng phá vỡ những ý niệm cố hữu trong lĩnh vực nghiên cứu hiện tại. Vì thế, nghiên cứu sinh luôn luôn phải nhớ rằng họ có thể dành thời gian để thỏa mãn trí trò mò của mình. Khám phá luôn bao gồm thất bại, nhưng điều đó là cần thiết và là nền tảng trong khoa học, kết quả là những khám phá mới và ý tưởng mới sẽ được sinh ra.

\section{Quy tắc thứ 8: Lên kế hoạch cho sự nghiệp và công bố}

Khi nghiên cứu sinh đi đến cuối chương trình học PhD, họ sẽ có một tập các kĩ năng đa dạng. Điều đó cho phép các cá nhân này theo đuổi sư nghiệp ở các lĩnh vực thích hợp cũng như các lĩnh vực chung chung. Hơn nữa, nếu nghiên cứu sinh nhận diện được một kĩ năng cần thiết nhưng lại còn thiếu, một chương trình học tiến sĩ liên ngành sẽ cung cấp các cơ hội tốt để phát triển nội lực cá nhân. Do đó, nghiên cứu sinh nên sớm nghĩ về nơi họ muốn kết thúc. Chú ý rằng sẽ có những yêu cầu khác nhau phụ thuộc vào con đường phát triển sau này mà nghiên cứu sinh muốn theo đuổi.

Một nhân tố quan trọng trong sự nghiệp là danh sách công bố. Các con đường phát triển sự nghiệp khác nhau có thể yêu cầu công bố trong các tạp chí khoa học khác nhau. Do đó, nghiên cứu sinh cần chọn tạp chí công bố cẩn thận. Làm việc trong môi trường liên ngành đồng nghĩa rằng nghiên cứu sinh sẽ có nhiều cơ hội lựa chọn để chọn các tạp chí hơn là nghiên cứu sinh đơn ngành, những người có ít sự lựa chọn tạp chí hơn rất nhiều.

Chắc chắn rằng nghiên cứu sinh quen thuộc với các tạp chí liên quan. Điều này đặc biệt quan trọng khi nghiên cứu sinh cần công bố công trình trên các tạp chí mà giáo sư hướng dẫn chính chưa có kinh nghiệm. Để thu hẹp phạm vi tìm kiếm, nghiên cứu sinh có thể cần thảo luận thêm với đồng nghiệp cùng định hướng nghiên cứu và chọn các dịch vụ web hoặc dịch vụ chọn tạp chí phù hợp. Từ đó, nghiên cứu sinh sẽ tìm được tạp chí phù hợp dựa trên thông tin về tiêu đề, phần tóm tắt, và các từ khóa. Nghiên cứu sinh nên chọn tạp chí mà bài báo của nghiên cứu sinh sẽ được đọc nhiều nhất, nhưng nghiên cứu sinh cũng nên cân nhắc đến các tạp chí khác có chỉ số trích dẫn cao. Một khi nghiên cứu sinh định vị được các ứng cử viên, hãy cân nhắc mục đích, phạm vi, người đọc, lịch sử công bố của tạp chí đó để xem liệu có phù hợp với bản thảo không. Bên cạnh đó, nghiên cứu sinh cần đọc kĩ hướng dẫn về quy tắc của tạp chí, và thử nhận diện người đọc của tạp chí đó cũng như các tùy chọn truy cập. Cuối cùng, khi nghiên cứu sinh đã lựa chọn được tạp chí, họ có thể không cảm thấy đó là lựa chọn đúng đắn, nhưng hãy thử liên lạc với giáo sư hướng dẫn và giải thích lý do. Một chương trình học tiến sĩ liên ngành sẽ cung cấp nhiều cơ hội nghề nghiệp. Các kĩ năng cần thiết có thể nhận biết được trong một vài khu vực khác nhau. Nghiên cứu sinh cần thử định hình chương trình học PhD của mình, từ đó đặt mục tiêu công bố thích hợp.

\section{Quy tắc thứ 9: Điều chỉnh cho phù hợp với người nghe}

Các nhà khoa học liên ngành sẽ phải trình bày công trình hiện tại tại các buổi gặp mặt, seminar, và hội nghị. Luận án của một nghiên cứu sinh có lẽ sẽ được thẩm định bởi các chuyên gia từ các ngành khoa học khác nhau. Do đó, nghiên cứu sinh cần điều chỉnh ngôn ngữ của mình sao cho thích hợp. Người nghe có thể bao gồm chuyên gia trong lĩnh vực của nghiên cứu sinh, cũng có thể có nền tảng trộn lẫn, hoặc không có nền tảng nghiên cứu. Nghiên cứu sinh cần thường xuyên thay đổi cách trình bày để phù hợp với đối tượng người nghe đang hướng tới. Bên cạnh đó, nghiên cứu sinh cũng phải trình bày bài báo cáo cho nhiều giáo sư hướng dẫn và những người cộng tác. Vì thế, nghiên cứu sinh cần đảm bảo đủ thời gian cho hoạt động này.

Trong một nhóm có nền tảng khoa học trộn lẫn hoặc một nhóm có nền tảng khoa học khác nhau, nghiên cứu sinh cần học cách dự đoán được lỗ hổng trong tri thức của mọi người. Từ đó, nghiên cứu sinh cần tạo ra một bài trình bày giống như một câu chuyện có tính chất kết dính mà ai cũng có thể hiểu được. Các khái niệm nên được trình bài từ nhiều góc độ khác nhau để nhiều đối tượng từ các nền tảng khác nhau hiểu được. Khi mô tả một khái niệm kĩ thuật hoặc khái niệm trừu tượng, nghiên cứu sinh cần tránh sử dụng biệt ngữ và nên sử dụng hình ảnh minh họa khi có thể. Hãy tưởng tượng rằng nghiên cứu sinh đang giải thích cho những người nghe không chuyên.

Khi được hỏi làm sao để tham gia trong các cuộc thảo luận khoa học và sự kiện kết nối công chúng, một phương pháp tiếp cận khác nên được triển khai. Nghiên cứu sinh nên sử dụng góc nhìn của một chú chim, đưa ra các ví dụ thú vị, và sử dụng ngôn ngữ rõ ràng. Nghiên cứu sinh không nên đánh giá thấp công việc của mình. Khi bị nghi ngờ về kết quả nghiên cứu, nghiên cứu sinh cần bắt đầu giải thích từ đơn giản nhất có thể đến phức tạp, và sau đó trau chuốt nếu cần thiết. Đối với sự kiện có tính chất công chúng, nghiên cứu sinh cần thực hành trước những độc giả thân thiện trước để nhận được phản hồi.

\section{Quy tắc thứ 10: Hãy nghỉ ngơi và thưởng thức cuộc sống}

Nghiên cứu sinh cần nhớ rằng, học PhD là một chặng dường dài chứ không phải chỉ là giai đoạn về đích. Hãy đảm bảo rằng bản thân sẽ luôn cảm thấy thoải mái với dự án mà nghiên cứu sinh đang tham gia. Những công bố khoa học tốt chỉ có một chút lợi ích nếu nghiên cứu sinh không tận hưởng giai đoạn học PhD. May mắn là, một chương trình nghiên cứu liên ngành thường cung cấp các cơ hội quý giá để một nghiên cứu sinh tìm ra được chủ đề nghiên cứu mà họ đang quan tâm. Vì thế, nghiên cứu sinh hãy tận dụng sự linh động và định hình công việc nghiên cứu sao cho họ cảm thấy thoải mái.

Đối với nhiều nhà khoa học, sự sáng tạo và hiệu suất làm việc thường cao nhất trong những ngày đầu tiên sau kì nghỉ. Nghiên cứu sinh hãy tận dụng các nguồn tài nguyên sẵn có từ phía giáo sư hướng dẫn và nhà trường, tham gia vào các câu lạc bộ và các hoạt động khác ngoài phạm vi nghiên cứu, và đi du lịch khi họ thực sự cần. Dành cho mình khoảng thời gian trống sẽ mang lại nhiều lợi ích về mặt dài hạn, đặc biệt khi nghiên cứu sinh trở lại công việc nghiên cứu với một tâm thế thoải mái và một tinh thần tràn đầy năng lượng.

Một chương trình PhD là một khoản đầu tư rất lớn về thời gian, năng lượng, và sự sáng tạo. Bằng cách áp dụng 10 nguyên tắc đơn giản vừa nêu, nghiên cứu sinh sẽ thành công trong chương trình học tiến sĩ liên ngành.

\begin{thebibliography}{00}

\bibitem{1} National Science Foundation Research Traineeship (NRT) Program - Program Solicitation. NSF; 2015 Oct. Report No.: nsf16503.

\bibitem{2} Jennifer Carney, Alina Martinez, John Dreier, Kristen Neishi, Amanda Parsad. Evaluation of the National Science Foundation’s Integrative Graduate Education and Research Traineeship Program (IGERT): Follow-up Study of IGERT Graduates. National Science Foundation; 2011. Report No.: NSFDACS06D1412.

\bibitem{3} Horizon 2020—Work Programme 2016–2017–3. Marie Sklodowska-Curie Actions. European Commission; 2016 Mar. Report No.: C(2016)1349.

\bibitem{4} Research Performance and Economic Impact Report 2013/14 [Internet]. EPSRC; 2014. https://www.epsrc.ac.uk/newsevents/pubs/economicimpactreport1314/.

\bibitem{5} Choi BCK, Pak AWP. Multidisciplinarity, interdisciplinarity and transdisciplinarity in health research, services, education and policy: 1. Definitions, objectives, and evidence of effectiveness. Clin Invest Med. 2006;29: 351–364. pmid:17330451

\bibitem{6} Gu J, Bourne PE. Ten simple rules for graduate students. PLoS Comput Biol. 2007;3: e229. pmid:18052537

\bibitem{7} Van Noorden R. Interdisciplinary research by the numbers. Nature. 2015;525: 306–307. pmid:26381967

\bibitem{8} Marino J, Stefan MI, Blackford S. Ten simple rules for finishing your PhD. PLoS Comput Biol. 2014;10: e1003954. pmid:25474445

\bibitem{9} Thessen A, Anne T, David P. Data issues in the life sciences. Zookeys. 2011;150: 15–51.

\bibitem{10} Knapp B, Bardenet R, Bernabeu MO, Bordas R, Bruna M, Calderhead B, et al. Ten simple rules for a successful cross-disciplinary collaboration. PLoS Comput Biol. 2015;11: e1004214. pmid:25928184
	
\end{thebibliography}


\pagebreak

\section{Phụ lục}

Bản gốc:

\paragraph{Introduction}

Many of today’s pressing research challenges require a multifaceted approach that combines several historically distinct disciplines. As a result, there has been a surge in funding for interdisciplinary PhD programmes. Some examples include the United States National Science Foundation (NSF) Research Traineeship (NRT) [1] (succeeds the Integrative Graduate Education and Research Traineeship [IGERT] [2]); the European Research Council’s Innovative Training Network (ITN) [3]; and, in the United Kingdom, strong growth in interdisciplinary doctoral programmes across all research councils, led by the Engineering and Physical Sciences Research Council (EPSRC) and their Centres of Doctoral Training (CDTs), with the strong support of UK universities and industrial partners [4].

First and foremost, an interdisciplinary PhD is a great chance for students to pursue truly novel research, a range of different career paths, and a stimulating intellectual life. However, these benefits are often accompanied by additional academic and logistical challenges. The rules presented here aim to provide guidelines that will enable PhD candidates to maximise the benefits of interdisciplinary research whilst minimising any burdens.

The term “interdisciplinary” itself has many different meanings in common usage; for the purposes of this article, we define “interdisciplinary” as the synthesis of 2 or more disciplines, establishing a new level of discourse and integration of knowledge [5]. Whilst most of the advice for students considering interdisciplinary programmes is similar to that of traditional graduate programmes [6], there are also differences that students should be aware of and prepare for; the interdisciplinary PhD programmes described above, and interdisciplinary research in general [7], bring unique opportunities as well as challenges.

There are several different types of interdisciplinary PhD programmes, and their organisation varies widely from country to country. In the UK, interdisciplinary programmes are increasingly funded through CDTs, as mentioned above; investigation of the major funding bodies in your field is therefore a good place to start for discovering which programmes are available. Different interdisciplinary PhD programmes may also be organised very differently: courses might be very structured, including preparation courses and short rotation projects before the PhD; some might include continuous training throughout the PhD; and some may require you to teach, while others may not. Since interdisciplinary programmes are by their nature highly varied, some also allow you to start the programme without first having chosen a supervisor; this enables you to familiarise yourself with different fields before choosing a PhD topic. For this type of programme, it is less important to have a particular subject area in mind. However, if you do, it is important to investigate the research groups associated with the program before you apply to ensure that the course is compatible with your aims.

We are a group of PhD students and programme directors at the Doctoral Training Centre (DTC), University of Oxford. The Oxford DTC was founded in 2002 and has since accommodated over 550 students across 7 interdisciplinary programmes, namely: EPSRC Life Sciences Interface, EPSRC Systems Biology, EPSRC and Medical Research Council (MRC) System Approaches to Biomedical Science, EPSRC and Biotechnology and Biological Sciences Research Council (BBSRC) Synthetic Biology, EPSRC Synthesis for Biology \& Medicine, EPSRC and MRC Biomedical Imaging, and BBSRC Doctoral Training Partnership. Drawing from our collective experience at the DTC, we present 10 simple rules for surviving and thriving in an interdisciplinary PhD. We highlight the importance of having a comprehensive plan with a realistic time line, maintaining good communication between all supervisors and collaborators, and making the most of the intellectual freedom that you are provided with when working in an interdisciplinary setting.

\paragraph{Rule 1: Involve everyone in the planning, and make contingency plans}

An interdisciplinary PhD will most likely require acquiring a wide range of new skills, involve more than one supervisor, and depend on multiple collaborators. It is therefore imperative to include all your supervisors early on when discussing and deciding the goals and aims of the PhD. Likewise, you should collectively lay out the plan to achieve those goals. This way, the expectations and requirements of everyone are more effectively managed (see Rule 2).

As with all PhDs, it is beneficial to generate a time line with specific milestones. This will allow you to assess the progress you have made throughout your PhD and identify any potential problems. Keep in mind that the PhD plan will change with time, so allow yourself room to manoeuvre. Importantly, make sure you have a contingency plan that covers eventualities such as collaborators not delivering data or dropping out entirely.

When planning your project, try to look ahead in terms of getting the right mentors and support network (see Rule 5). Ask your supervisors to make introductions; this can make it much easier to meet people, especially at the beginning of the PhD when you are still finding your feet.

Finally, ensure that you understand your administrative obligations. Depending on your institution and department, there may be different requirements for each stage of progression in a PhD. While the high degree of flexibility may be beneficial and allow you to choose the requirements that suit you best, it also means that you are more likely to fall outside of the normal procedures. As a result, the administration is not always prepared for the requirements of interdisciplinary students. It is therefore, as always, important to try to plan your last year in advance [8]. This will reduce the number of times you run into unexpected hurdles and make the final months less daunting. Take initiative and think ahead.

\paragraph{Rule 2: Be a diplomat: Start managing expectations early on}

Try to organise regular meetings with all your supervisors and collaborators. As well as ensuring progression of the PhD, collective planning and discussion helps to prevent frustrating situations and disagreements. Recognise that supervisors from different fields may have very different expectations for what is achievable in a particular time frame or may find it hard to judge the difficulty and likely time span of research outside their own area of expertise.

If questions regarding the direction of the PhD arise, such as the best approach to a problem, it is normally best to discuss the problem directly with all participating parties in the same room.

With several parties involved in your PhD, it is essential to keep communicating on a regular basis, with regular time slots for video conferences or face-to-face meetings.

\paragraph{Rule 3: Define the boundaries of your research: Explore and familiarise, then be pragmatic}

Once everyone is on the same page and you have laid out a plan for your PhD, you need to start to do your research. A “traditional” PhD student quickly develops very deep knowledge of a narrow subject area in a particular discipline, whereas an interdisciplinary student is likely to obtain knowledge that is less deep but spread out across several subject areas and multiple disciplines. It is therefore important to anticipate and explore the fields relevant to your PhD early on. Attending a wide range of seminars or even undergraduate lectures is a good way to gain a foundation of understanding in a new field and to learn discipline-specific terminology. Time spent investigating these complementary subject areas early will be beneficial in the long run, as it will enable you to see the bigger picture and place your work in context.

Furthermore, the quality, quantity, and structure of data vary between disciplines [9]. Make sure you know what to expect, and perform “sanity checks” on the data before you use it for anything. This will enable you to identify any issues and allow you to take the necessary steps to interpret the data correctly [10].

Whilst it is important to become familiar with all the relevant topics across the disciplines of your work, you cannot learn everything. As the PhD progresses, start focusing on the core areas that are directly relevant to your research. Consider a funnel-shaped learning approach, where you learn the fundamentals of the field first and narrow down on more specialised topics at later stages. Explore and familiarise yourself, but then try to be pragmatic and goal oriented.

\paragraph{Rule 4: Don’t be embarrassed: Always ask the “stupid question”}

When you are exploring new fields, it is normal to feel estranged, alone, and lost. Seminars, group meetings, and research talks in this alien field can seem like they are being conducted in an entirely different language. It is not uncommon to attend meetings where unfamiliar jargon is heavily used. Consider immersing yourself and “spending time with the natives” as much as possible; this is often the quickest and most effective way to become proficient in the new field.

Being in between disciplines naturally means that a lot of time will be spent being a novice, and progress may feel slow at the beginning. You will always have to ask a lot of questions, and it is fine to solicit help shamelessly. You might feel like your questions are too simple to waste anyone’s time with, but keep in mind that most people are happy to see you engaging with their subject. Furthermore, questions from people with a different background often lead to a new perspective that might not have been considered otherwise.

Do not be afraid to ask the “stupid” question; what seems trivial might not be quite so simple.

\paragraph{Rule 5: Build a network: Find other people to complement the gaps}

Due to the nature of interdisciplinary research, there will always be significant gaps in your knowledge. Identify fellow researchers early on in your PhD who complement your knowledge base; you will then be able to call upon each other’s expertise when required. You might not find all the help you need in your department. A quick internet search and an email may help you to find the right group; most scientists are friendly and happy to share their knowledge. If in doubt, your supervisors should be able to put you in contact with the relevant people.

Fostering and maintaining relationships with researchers from diverse backgrounds is a key aspect of doing an interdisciplinary PhD; it will make your life a lot easier when it comes to finding someone to answer your questions, finding suitable collaborations, or getting your hands on that crucial data set. Try to find people with whom you enjoy working and who are good at what they do. It is worth spending some time on developing good relationships, as there is a good chance you will keep working together beyond the PhD. A paper specifically focusing on different aspects of cross-disciplinary collaborations is available in the 10 simple rules series [10].

When communicating with researchers in different disciplines, be sure to clarify what people mean by the terminology they use, as the same word may mean different things in different fields. As an example, consider the word “orthogonal”: in geometry, 2 lines are orthogonal if they form a right angle; in statistics, independent variables that affect a dependent variable are considered to be orthogonal if they are uncorrelated; in taxonomy, a classification is orthogonal if each item is a member of only 1 group; and in biochemistry, the 2 types of DNA base pairs are considered orthogonal interactions. Ask for and give explanations for technical jargon, as the language barrier is always present in interdisciplinary collaborations.

While it is good to have some close contacts, it is also valuable to develop a more diverse and loosely connected network. You might consider this a “dormant” network that you can dip into when the opportunity or the need arises; it may just lead to a fruitful collaboration.

Many of the most useful conversations happen when you least expect it. Try to socialise with your peers, especially in nonacademic settings.

\paragraph{Rule 6: Embrace your unique skillset and use it to redefine discipline boundaries}

The value of symbiotic relationships between disciplines is well recognised. It comes from the realisation that some research questions are neglected because they do not fit inside the traditional boundaries of the discipline. Similarly, suitable methods or techniques for a particular problem might already exist in another field but have gone undetected due to the isolation of disciplines. As an interdisciplinary researcher, you are well placed to tackle those neglected problems. By embracing your unique skillset and looking for opportunities to connect the dots, you may find yourself redefining traditional boundaries or facilitating a groundbreaking translation of technology from one discipline into another.

This also holds true in collaborations. Often, solutions to problems that you would consider trivial are of high value to scientists in other fields. Recognise these opportunities and be open about your competencies, even the simplest ones. You may be surprised how much you can contribute. This is a good way to gain visibility among your colleagues and open new opportunities.

However, be careful, as it is easy to become a service provider to others in a different field when you have skills that they do not. You may not want to become the IT administrator responsible for routine data processing and organisation or the lab assistant responsible for routine and tedious experiments. Only consider doing these things if they have clear limits and clear benefits.

\paragraph{Rule 7: Feel free to swim against the flow, to experiment, and to fail}

Established fields will commonly have a dogmatic approach to certain problems that has evolved over generations of researchers. These well-established methods have been proven to work and have a vast amount of research to back them up. However, as a researcher with a potentially different view of the problems, you may be attracted towards approaches that are considered unconventional. Do not be afraid to challenge the dogma of the field, provided your approach has previously unanticipated benefits.

Furthermore, as someone who sits between disciplines, the techniques you require may not have been developed yet or may simply never have been applied to an individual discipline before. Developing new techniques (or applying “foreign” techniques) and proving their utility comes with higher risks and potentially higher rewards. As the forerunner, you will have to figure things out for yourself. You will get stuck and your approach may fail. Do not lose motivation during these seemingly unproductive phases. You may even encounter resistance to breaking dogmas from your supervisors, so it is important to remember that you can afford to spend time on satisfying your curiosity. Exploration involves failure, but it is fundamental and necessary in science, as it results in new discoveries and ideas.

\paragraph{Rule 8: Plan your career and publish accordingly}

By the end of the PhD, you will have a wide set of transferrable skills. These allow you to pursue careers in niche areas as well as more general fields. Furthermore, if you identify a skill that you do not yet have, an interdisciplinary PhD is a good opportunity to build the required competencies—so start thinking early about where you want to end up. Note that there will be different requirements depending on which career path you choose to follow. There are many places your skills could take you.

Another important factor in your career will be your publication record. Different career paths may require publishing in different journals, so you need to choose carefully. Working in an interdisciplinary setting means that you have a much larger variety of journals to choose from than a single-subject PhD, where there are usually only a select few journals.

Make sure you familiarise yourself with all relevant journals. This is especially important when you publish in journals with which your main supervisor has no experience. To narrow down the search, you can discuss options with colleagues in your field(s) and choose from a number of web services and journal selectors that find suitable journals based on information from your title, abstract, and other key words. The natural place to publish will most likely be the journals whose papers you most frequently read, but also look at the journals where these papers are frequently cited. Once you have identified some candidates, consider if the aims, scope, readership, and publication history of the journal are in alignment with your manuscript. Also, familiarise yourself with the rules of the journal by reading the author guidelines, and try to identify its readership and access options. When deciding on a journal, you may not feel that it is your choice, but try to engage in the discussion with your supervisors and explain your reasoning. An interdisciplinary PhD provides many career opportunities, as your unique skill set may be appreciated in a number of different sectors. Try to shape your PhD to your career plans and aim to publish accordingly.

\paragraph{Rule 9: Adjust to your audience}

Interdisciplinary researchers will have to present work at meetings, seminars, and conferences with vastly different foci (e.g., to both theoretical and experimental audiences). Your thesis will probably be examined by experts from different fields. Thus, you will have to adjust your language appropriately; your audience may comprise specialists in a particular field, those with mixed backgrounds, or those with no scientific training at all. This is, of course, applicable to all researchers, but the frequency of having to change a presentation to suit an audience is increased in interdisciplinary areas. Also, remember that you may have to run your proposed presentation past multiple supervisors and collaborators, so make sure to allocate enough time for this.

For a group of mixed backgrounds or a group with a different background, it is essential that you learn to anticipate any gaps in people’s knowledge and provide a coherent story no matter how basic you think the material is. It may also be a good idea to explain concepts from several angles to accommodate as many people as possible. When describing an abstract or technical concept, avoid jargon and use visual representations where possible. Imagine explaining it to your nonexpert self before you started doing your research.

When asked to participate in science communication and public engagement events, another approach is required. A bird’s-eye view, interesting examples, and clear language are crucial for effective communication. Do not underestimate the work that goes into these presentations; you will have to think about your research from a completely different perspective to your own. When in doubt, always start explaining things in the simplest terms possible and elaborate later on, if necessary. For high-profile events or key interview presentations, always practice in front of a friendly audience and act on feedback.

\paragraph{Rule 10: Relax and enjoy}

A PhD is a marathon, not a sprint. Make sure you are comfortable with your project and that you are in a position to enjoy the experience. The best publication record will be of little benefit if you do not enjoy the process. Fortunately, an interdisciplinary PhD often provides unique opportunities for you to design your research around your interests. Use this flexibility and shape your work into something that you enjoy and fully embrace.

For many scientists, creativity and productivity are highest during the first days after a break. Make use of your supervisors’ and the university’s resources, engage in clubs and other activities outside of your research, and take vacations when you need them. Taking time off will benefit you in the long run, as you will return to your research with renewed energy and a fresh mind.

A PhD is a huge investment of your time, energy, and creativity—the finale of years of education and training. By following these 10 simple rules, it should be a rewarding and empowering experience for you!

\end{document}          
